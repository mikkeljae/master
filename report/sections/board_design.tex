%!TEX root = ../main.tex
\section{Board Design} % (fold)
\label{sec:board_design}
Multiple electronics components are required to realise the double pendulum being developed throughout this report.
This section explores the design of those components and the communication between them.
Most of the circuitry is combined on a single PCB, the driver board, however some of the functionality is moved to smaller boards, local to the place where they are needed.\\

The components of the system are listed here:
\begin{itemize}
	\item Motor Encoder
	\item Joint Encoder
	\item End Stops
	\item Motor Driver
	\item Emergency Circuitry
	\item Relay Driver
	\item RF Transceiver
	\item MicroZed 
\end{itemize}
Many of these components require a number of subcomponents, which will be explored further in later sections.

\subsection{Voltage Rails} % (fold)
\label{sub:voltage_rails}
In the design of the board it is necessary to determine which voltage rails are required for the system to function.
The power delivery for the MicroZed was designed by the authors in an earlier project \cite{isaswarm} and will be reused with minor changes.
This power delivery system provides, amongst other voltages, the 3.3V rail, originally intended for powering the MicroZed IO banks only.
In this system however, the RF tranceiver is also powered from this rail.
As a result a review of the circuitry around the 3.3V rail is necessary to ensure that it can provide the required power.
The MicroZed power delivery system, the encoders of the system, the endstops and part of the emergency circuitry are all driven from a 5V rail.
The Motor Driver, the HIP4081 \cite{driver} and the relay driving circuitry is driven from a 12V rail.
Finally, the motor is driven from a 24V rail, which will also be the main supply for the system.
This rail is not crucial to the design of the board as it is provided by an external, mains connected power supply and will not be explored further in this section.\\
The current requirement for each of these rails are discussed in the following paragraphs

\paragraph{3.3V:} % (fold)
\label{par:3_3v}
As mentioned, this rail is intended for powering the MicroZed IO banks.
In the original design the LMR10510XMF DC/DC converter is used to provide the necessary power.
This chip is capable of supplying up to 1A.
Components exist in the same series which are pin compatible and are capable of supplying up to 2A.
The RF Transceiver used in this project, however, \ref{} requires only up to 15mA when receiving data.
Assuming that Avnet has already provided headroom for the IO bank supply and considering that this project makes little use of the IO on the Zynq-7000 chip, it is deemed unnecessary to upgrade this supply and the original design is used as is.
% paragraph 3_3v_ (end)

\paragraph{5V:} % (fold)
\label{par:5v}
This rail supplies, mainly, the MicroZed.
The authors previously found \cite{isaswarm} that the maximum expected current draw seen from the MicroZed is 1.85A at 5V.
In addition, in this design, the 5V rail also powers the encoders, endstops and emergency circuitry.
There are two types of encoders, the HEDS-5540 \cite{heds5540}, which requires a maximum of 85mA and a Rolin magnetic encoder, which requires a maximum of 25mA.\\
The endstops are realised using the TCST2103 infrared sensor \cite{tcst2103}.
The majority of the current supplied to this sensor is used to power the infrared LED present in the component.
This current is decided by the resistor put in series with the LED and is estimated to be around 50mA per LED for a total of 100mA.
Another 10mA is added to that figure due to the collector current possible on the transistor side.\\
Finally the emergency circuitry along with some other digital electronics are powered from the 5V rail.
As these are all digital IC's that operate on nothing but signals, their individual powers are negligible but a very conservative 25mA power budget is provided for all of the digital electronics on the 5V rail. This brings the total power budget for the 5V rail to $\approx$2.1A.
See table \ref{tab:5vpowerbudget} for a full overview.

\begin{table}
	\centering
	\begin{tabular}{l|r}
		 Component & Current [mA]\\
		 \hline
		 MicroZed & 1850\\
		 HEDS-5540 & 85\\
		 Rolin Enc. & 25\\
		 TCST2103 & 110\\
		 Digital & 25\\
		 \hline
		 Total & 2095
	\end{tabular}
	\caption{Power budget for the 5V rail}
	\label{tab:5vpowerbudget}
\end{table}

In \cite{isaswarm} the authors used a design in which the PTH08080 is used to generate a 5V rail.
Reusing this module will save on design time as well as the budget available to the project and as such is desirable.
This module however, is capable of supplying only 2A at 5V, slightly less than the value calculated in this section.
By far the largest contributor to the power budget is the MicroZed.
The calculation of the contribution from the MicroZed is done assuming 85\% utilisation of PL and a conservative 80\% efficiency of internal DC/DC converters.
Considering this, it is safe to assume that the real power draw from the MicroZed is significantly smaller than the calculated value and as a result it is chosen to reuse the PTH08080 despite of apparent shortcomings of the module.
% paragraph 5 (end)

\paragraph{12V:} % (fold)
\label{par:12v}
This rail powers the HIP4081 motor driver, the relay coils, the bootstrap circuitry and the 5V DC/DC converter.
The PTH08080 DC/DC converter has a maximum input voltage of 18V and must be powered from the 12V rail rather than the 24V rail.
The module can provide 2A at 5V and as such will require $\approx$0.85A from the 12V rail.\\
There are two relays in the design, a smaller relay for controlling the inrush current and the larger main supply relay.
Both of these require power to stay in the closed position.
The first, the G6B \cite{g6b}, requires 16.7mA while the former, the LEV100A4ANG \cite{lev100}, requires 461mA.\\
The motor driver of the system, the HIP4081 requires only 10mA.
As mentioned, the bootstrap circuitry is also powered from the 12V rail.
Rather than a steady supply, this circuitry requires a large peak current for short periods while charging in between switching.
For this reason the design choice here is to determine what is available and choose the component which yields the most headroom while still being economically feasible.
Amongst all of the components the total power budget for the 12V rail is $\approx$1340mA.
See table \ref{tab:12vpowerbudget} for a full overview.

\begin{table}
	\centering
	\begin{tabular}{l|r}
		 Component & Current [mA]\\
		 \hline
		 PTH08080 & 850\\
		 G6B & 16.7\\
		 LEV100A4ANG & 461\\
		 HIP4081 & 10\\
		 \hline
		 Total & 1337.7
	\end{tabular}
	\caption{Power budget for the 12V rail}
	\label{tab:12vpowerbudget}
\end{table}

The PTN78020 delivers 6A at 12V and is part of the same series as the PTH08080 described earlier.
This allows many of the same design procedures to be reused.
In addition, the 6A current limit leaves sufficient room for the current spikes expected from the bootstrap circuitry.
The bootstrap circuitry is modified slightly to accomodate the limited supply as described in section \ref{}
% paragraph 12v_ (end)

% subsection voltage_rails (end)

% section board_design (end)