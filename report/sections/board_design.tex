%!TEX root = ../main.tex
\section{Board Design} % (fold)
\label{sec:board_design}
Multiple electronics components are required to realise the double pendulum being developed throughout this report.
This section explores the design of those components and the communication between them.
Most of the circuitry is combined on a single PCB, the driver board, however some of the functionality is moved to smaller boards, local to the place where they are needed.\\

The components of the system are listed here:
\begin{itemize}
	\item Motor Encoder
	\item Joint Encoder
	\item End Stops
	\item Motor Driver
	\item Emergency Circuitry
	\item Relay Driver
	\item RF Transceiver
	\item MicroZed 
\end{itemize}
Many of these components require a number of subcomponents, which will be explored further in later sections.

\subsection{Voltage Rails} % (fold)
\label{sub:voltage_rails}
In the design of the board it is necessary to determine which voltage rails are required for the system to function.
The power delivery for the MicroZed was designed by the authors in an earlier project \cite{isaswarm} and will be reused with minor changes.
This power delivery system provides, amongst other voltages, the 3.3V rail, originally intended for powering the MicroZed IO banks only.
In this system however, the RF tranceiver is also powered from this rail.
As a result a review of the circuitry around the 3.3V rail is necessary to ensure that it can provide the required power.
The MicroZed power delivery system, the encoders of the system, the endstops and part of the emergency circuitry are all driven from a 5V rail.
The Motor Driver, the HIP4081 and the relay driving circuitry is driven from a 12V rail.
Finally, the motor is driven from a 24V rail, which will also be the main supply for the system.\\
In summary, the rails used in this system are:
\begin{itemize}
	\item 3.3V
	\item 5V
	\item 12V
	\item 24V
\end{itemize}

% subsection voltage_rails (end)

% section board_design (end)