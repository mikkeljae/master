%!TEX root = ../main.tex
\section{Preliminary Testing} % (fold)
\label{sec:preliminary_testing}
In an effort to split the development into smaller, more managable parts, it was decided to initially develop the system such that it fulfills the following requirements:

\begin{itemize}
	\item The cart should be able to move with known position.
	\item Endstops on the platform should stop movement.
	\item Emergency button should stop movent.
\end{itemize}

Clearly, making the cart move is crucial in the development of system and forms the basis for any later developments.
The latter two requirements are related to the safe operation of the platform.
It is expected that at least during development, an error may occur that causes the cart to move uncontrollably.
In such a situation the programmer should be able to completely shut off power to the motor.
The endstops will ensure that a minimum of mechanical or electrical damage is incurred on the platform in case of a cart crash.
The following paragraphs will explore what is required in order to fulfill the above requirements.
\\~\\
Safety first.
A safety system should, whenever possible, be isolated from the remaining system.
That is, it should depend on non of the remaining circuitry or programming.
Usually, if it can at all be avoided, no programming should be involved in determining a safety condition.
With this in mind, the safety system for this platform should be designed in such a way that, when activated, it will cut power to the motors.
In order to more easily identify the cause of the fault, the remaining electronics should remain operational to maintain the current program status.
\\~\\
The first requirement requires a few tasks to be solved.
Firstly it is necessary to device some form of driving circuitry for the motor driving the timing belt.
The motor should be able to move in either direction.
As the Maxon 148867 DC Motor is a brushed DC motor the standard circuit for achieving this behaviour is an H-bridge.
Initially an H-bridge will be designed along with the required driving circuitry for that bridge.
In addition to the above, it is required that the position of the cart is known.
The Maxon motor is equipped with a relative encoder with 500 counts per revolution.
A method for determining the absolute position will have to be created.
% section preliminary_testing (end)