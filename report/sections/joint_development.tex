%!TEX root = ../main.tex
\section{Joint Development}
\label{sec:joint_development}


\subsection{Joint Microcontroller}
\label{sub:joint_microcontroller}


\mikkel{Not finished}
\begin{itemize}
	\item Low power
	\item Compact size
	\item Three interrupt pins
	\item Eight I/O pins in total 
	\item SPI interface
	\item Non volatile memory
	\item 3.3V/5V supply
\end{itemize}


\begin{table}
	\centering
	\begin{tabular}{l}
		 \textbf{Requirements:} \\ \hline
		 Low power \\ \hline
		 Compact size \\ \hline
		 Three interrupt pins \\ \hline
		 Eight I/O pins in total\\ \hline
		 SPI interface \\ \hline
		 Non volatile memory \\ \hline
		 3.3V/5V supply \\ \hline
	\end{tabular}
	\caption{Hmmm. Not satisfied with this.}
	\label{tab:joint_mic_rec}
\end{table}

We chose the the ATtiny84 - yay!


\subsection{Voltage Rails}
\label{sub:voltage_rail}
It is necessary to determine which voltage rails are needed for the components on the joint board to function.
The wireless module, \texttt{NRF24}, and the ATtiny84 can both be run at 3.3V, but the Rolin encoder needs 5V.
Clearly a 3.3V rail and a 5V rail is needed. 
The current consumption on both rails need to be determined in order to specify the requirements for the voltage regulators and the required size of the battery needed.
\mikkel{There should be a nice introduction to this section about the Joint in general... Battery and so on.}


\paragraph{3.3V:}
\label{par:3_3v}
The \texttt{NRF24} module has maximum supply current of 12.3mA \cite{NFR24L01} and the \texttt{ATtiny82} has a maximum of 9mA \cite{attiny84}.
This totals 21.3mA of current that the 3.3V rail needs to supply.

\paragraph{5V:}
\label{par:5v}
The only component drawing current from the 5V rail is the Rolin encoder which has a maximum supply current of 20mA \cite{RLBD01}.

\subsection{Power Consumption} 
\label{subsub:power_consumption}
The power consumption at each rail is calculated and 

\begin{table}
	\centering
	\begin{tabular}{l|r|r}
		 Voltage rail 	& Current [mA] 	& Power [mW]\\
		 \hline
		 3.3V 			& 21.3			&70.3		\\
		 5V  			& 20 			&100		\\
		 \hline
		 Total 			& 				&170.3		
	\end{tabular}
	\caption{Power usage on voltage rails and in total.}
	\label{tab:power_joint}
\end{table}

\subsection{Battery}
\label{subsub:battery}
The physical dimensions of the battery is constrained by the free space in the designed joint.
It needs to be able to fit in a cylinder with a diameter of 57mm and height of 9mm.
Furthermore needs to have enough capacity to power the circuit for a full workday of a student, which is estimated to be 10 hours.
\mikkel{10 hours workday?}

\paragraph{9V Battery}
Batteries in the standard 9V package can purchased with with enough capacity, but the physical size disqualifies it.

\paragraph{Button Cell Battery}
Button cell batteries does live up to the physical constraints, but does not have the wanted capacity. 
Several of them can be put in series, but they still does not yield the wanted capacity.

\paragraph{Li-Ion Battery}
Lithium-Ion batteries have a very high energy density and it should therefore be possible to find a battery that fit both the physical and capacity constrains.
WE FOUND ONE!!!!!