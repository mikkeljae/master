%!TEX root = ../main.tex
\subsection{Analysis} % (fold)
\label{sub:analysis}

In section \ref{sec:analysis} a number of requirements were found relating to the design of the joints.
This list briefly summarizes these requirements

\begin{itemize}
	\item An encoder should be implemented to enable tracking the relative position of the joints.
	\item Data should be transmitted from the joint to the controller.
	\item The joints should be significantly heavier than the pendulum arms to accomodate the point-mass assumption. 
\end{itemize}

These requirements contain both mechanical and electrical aspects.
The authors' main focus is on power and embedded design and as such the mechanical design is limited in comparison to the electrical design.
The mechanical design was developed with help from mechanical engineer at SDU: Jørgen Maagaard.

\subsubsection{Mechanics} % (fold)
\label{ssub:mechanics}
While the current project requires only two joints, it is desirable to design the joint in such a way that it is easily expandable, allowing the addition of more joints at a later time.
In order to minimize influence on the control it is crucial to minimize friction in the joint.
For this reason, low-friction bearings are to be used in the joint.
% subsubsection mechanics (end)

\subsubsection{Encoder} % (fold)
\label{ssub:encoder}
A multitude of different encoder types exist on the market however one in particular was recommended to the authors.
Namely the Rolin Rotary Magnetic Encoders.

% subsubsection encoder (end)
% subsection analysis (end)