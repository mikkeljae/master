%!TEX root = ../main.tex
\subsection{Analysis} % (fold)
\label{sub:analysis}

In section \ref{sec:analysis} a number of requirements were found relating to the design of the joints.
This list briefly summarizes these requirements

\begin{itemize}
	\item An encoder should be implemented to enable tracking the relative position of the joints.
	\item Data should be transmitted from the joint to the controller.
	\item The joints should be significantly heavier than the pendulum arms to accomodate the point-mass assumption. 
\end{itemize}

These requirements contain both mechanical and electrical aspects.
The authors' main focus is on power and embedded design and as such the mechanical design is limited in comparison to the electrical design.
The mechanical design was developed with help from mechanical engineer at SDU: Jørgen Maagaard.

\subsubsection{Mechanics} % (fold)
\label{ssub:mechanics}
While the current project requires only two joints, it is desirable to design the joint in such a way that it is easily expandable, allowing the addition of more joints at a later time.
In order to minimize influence on the control it is crucial to minimize friction in the joint.
For this reason, low-friction bearings are to be used in the joint.
% subsubsection mechanics (end)

\subsubsection{Encoder} % (fold)
\label{ssub:encoder}
Each joint should implement an encoder to enable tracking the angular position of the joint.
The resolution of each joint should be sufficient to allow for proper control.
\thomas{Note on the required resolution}
In addition to resolution, ease of manufacture and design is also an important aspect.
Many encoders are developed for a specific mounting method or product, such as the HEDS-5540 used on the Maxon Motor.
One encoder in particular was recommended to the authors; the Rolin Rotary Magnetic Encoders.
This is a series of encoders which, as the name suggests, utilises magnetic fields in order to determine the angular position of the encoder.
They consist of a magnetic ring with a number of poles and a PCB sensor.
The magnetic ring is available in different sizes with different number of poles which, along with the type of PCB sensor decides the resolution of the encoder.
The specific encoder configuration used in this project allows for 8000 CPR or one count per 0.045\degree.
The encoder is incremental with an A, B and Z channel where A and B are 90\degree out of phase and Z is a unique reference mark.
Using the reference mark it is possible to maintain absolute knowledge of the angular position of the joint after the initial encounter with the reference mark.
This is only possible assuming that there is no drift or slippage in the system, a reasonable assumption considering the mechanical setup.
Communication with the encoder is done using the RS422 standard.
This standard describes a method of transmitting digital signals using differential signals and requires specialised hardware to translate between ordinary digital signals and RS422.

% subsubsection encoder (end)
\subsubsection{Electronics} % (fold)
\label{ssub:electronics}
As per the initial requirements listed in this section each joint should be able to track the position of the joint and communicate this information wirelessly to the main controller board.
In order to accomplish this task some form of microcontroller is required along with the power delivery and the RS422 receiver mentioned previously.
In addition to this the physical dimensions of all electronics should fit within the joint.
\paragraph{Power Delivery} % (fold)
\label{par:power_delivery}
The required voltage rails on the joint board are dictated by the RF module and the encoder.
The former requires 3.3V and the latter 5V.
Since the joints are to be wireless it is necessary to power them from a battery.
To avoid added complexity there will be no charging circuitry and charging the battery should be done externally to the joint.
A joint should last for as long as is possible while still fitting within the enclosure.
This requirement essentially narrows the battery choices to LiPo cells as these generally have the highest energy density of the commercially avaiable battery types.
1S LiPo cells are rated at 3.7V and so DC/DC converters are required to reach both rails, a buck converter for the 3.3V rail and a boost converter for the 5V rail.
In order to correctly dimension the converters it is necessary to determine the power draw from each rail:
\begin{itemize}
 	\item \textbf{3.3V:} This rail powers the NRF24 module which has a maximum supply current of 12.3mA \cite{NFR24L01} as well as the ATTiny84 with a maximum power draw of 9mA \cite{attiny84}.
 	This yields a total of 21.3mA at 3.3V or 70.3mW.
 	Additional digital circuitry placed on this rail is considered negligible in the power budget.
 	\item \textbf{5V:} This rail powers the encoder with a maximum supply current of 20mA \cite{RLBD01} as well as the RS422 receiver which has a supply current of 52mA \cite{rs422rec}.
 	This yields a total of 72mA at 5V or 360mW.
 	Additional digital circuitry placed on this rail is considered negligible in the power budget.
\end{itemize}
For a battery to have sufficient capacity to power the joint for a full workday, estimated at 10 hours, it should have at least $\approx4300$mWh
\thomas{What is the actual power draw from the rs422 receiver? 52mA seems excessive and results in approx 3 hours on-time for the joint.}
% paragraph power_delivery (end)
\paragraph{Microcontroller} % (fold)
\label{par:microcontroller}
The requirements for the microcontroller used are sparse as its main task is simply counting the current position and, occasionally, transmitting that position back to the controller board.
It should include support for SPI as well as three I/O pins which support interrupts for the A, B and Z channel of the encoder.
Since space is at a premium in the joint it should be compact in size and, finally, should be as low power as possible.
The ATTiny84 \cite{attiny84}
% paragraph microcontroller (end)
% subsubsection electronics (end)
% subsection analysis (end)