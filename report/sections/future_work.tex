%!TEX root = ../main.tex
\subsection{Future Work}
\label{sub:future_work}
As is customary with projects, this one did not fully accomplish what the authors originally intended.
This section will go through some of the work required to correct mistakes done by the authors as well as present some of the features that may be of interest for future developers.

\subsubsection{Corrections} % (fold)
\label{ssub:corrections}
A number of mistakes, mostly regarding the PCB design, were uncovered throughout the duration of the project.
Both the controller board and the joint board, while functional, would benefit from a second iteration to correct these mistakes.
\\~\\
Perhaps most notably the 4ms error uncovered in section \ref{subs:requirement_enum:motor_speed_direction} requires a thorough investigation so that the root cause of the issue can be found and corrected. 
\\~\\
It was found that the rod and joint mount, made of carbon fiber and 3D printed plastic repectively, was not sufficiently stiff to avoid excessive movement along the Z-axis of the system.
The authors invision an ISA (Individual Study Activity) for a mechanical engineering student who might be able to bring knowledge of material properties which the authors do not possess.
% subsubsection corrections (end)
\subsubsection{Linux Implementation} % (fold)
\label{ssub:linux_implementation}
The intent with the system is to make it controllable from Linux.
This can be done using different methods.
A kernel driver could be written such that a user can send commands to and receive data from the system simply by issuing read/write commands on a \texttt{tty} interface using userspace software.
\\~\\
Another approach involves setting up parallel processing on the two cores of the MicroZed.
One core running Linux and another maintaining communication with the inverted pendulum system.
Communication between the two can be done using direct memory access, DMA.
Using this method the strictly real-time tasks can be done on a bare-metal system.
% subsubsection linux_implementation (end)
\subsubsection{Determination of System Parameters} % (fold)
\label{ssub:determination_of_system_parameters}

% subsubsection determination_of_system_parameters (end)