
%!TEX root = ../main.tex
\section*{Preface}
\addcontentsline{toc}{section}{Preface}
This report is written in partial fulfilment to the requirements of a master thesis of 40 ECTS points at the University of Southern Denmark.
It is written as a part of the authors' education as electrical engineers at the Faculty of Engineering.
\\
Leon Bonde Larsen has been the the supervisor throughout the project.

\section*{Acknowledgment}
\addcontentsline{toc}{section}{Acknowledgment}

We would like to express our gratitude towards everyone that helped us throughout the duration of this thesis.
A special word of thank goes to Jesper Nielsen for his assistance and feedback on developing our PCB and Jørgen Maagaard for his help and ideas in relation to development of the joint mechanics. 

Thank you to Carsten Albertsen for educative discussions on electronics design, PCB layout and enjoyable lunch breaks. 
\\~\\
A special thank you goes to our supervisor Leon Bonde Larsen for his invaluable assistance and patience.
\\~\\
Finally, we would like to acknowledge that this project would not have been possible without the endless support of our friends and families.

\vspace{0.5cm}
\begin{center}
	\begin{minipage}[t]{.49\textwidth}\large
		\begin{center}
		Mikkel Skaarup Jaedicke\\
		\vspace{1cm}
		\hrule
		\vspace{0.5cm}
		Thomas Søndergaard Christensen
		\vspace{1cm}
		\hrule
		\end{center} 
	\end{minipage}
\end{center}

\vfill
  \begin{center}
    \textsl{The report, source code, data, plotting scripts and simulations can be found at:}  
    \end{center}
    \vspace{-5pt}
    \begin{center}
	\renewcommand{\UrlFont}{\color{black}\normalsize\tt}
    \url{github.com/mikkeljae/development_of_a_pendulum_control_system}
   \end{center}
\newpage

\section*{Abstract}
\addcontentsline{toc}{section}{Abstract}
This project is done in an effort to develop a platform for experimenting with control of unactuated systems.
An overview is given of unactuated systems in general and of the inverted pendulum in particular.
The inverted pendulum is analysed in an effort to determine the requirements of a system capable of controlling such a pendulum.
This report is the design document of an inverted double pendulum mounted on a cart is designed and implemented.
This includes designing a joint capable of wirelessly transmitting its angular position at regular interface to the main computational platform, a Zynq development platform, the MicroZed.
This choice meant the development of a PCB capable of correctly interfacing and powering the platform.
This PCB is also houses the driver of the motor on the inverted double pendulum system.
Throughout the work methods were developed to facilitate the correct development and verification of the PCBs used in the project.
Finally every requirement of the designed system is verified and evaluated upon.