%!TEX root = ../main.tex
\section{Conclusion} % (fold)
\label{sub:conclusion}
All of the requirements have been concluded upon throughout the report and  will not be repeated in full for this discussion.
Instead an overview of some of the accomplishments and shortcomings of the project will be presented.
The goal of the project was to develop a cart-mounted inverted double pendulum for experimenting with underactuated systems.
In order to do this an overview was gathered on the applications of underactuated systems, especially of similar pendulum systems.

The general design method used throughout the report is based on a workflow of analysis $>$ design $>$ implementation $>$ verification.
Adhering to this system not only provided the authors with a better design, but also with better documentation throughout the process.
\\~\\
Two PCBs were designed, the controller board and the joint board.
For the former a motor driver was designed and the power delivery of the MicroZed, designed by the authors in a previous project, was added to enable the use of a Zynq-7 series SoC for the main computational platform.
The joint board uses wireless communication in order to transmit the joint positions to the controller board.

These PCBs were designed following design steps determined by the authors.
It is believed that following these steps greatly decreased the number of errors in the final design.
They were, however, by no means error-free and would both benefit from a second iteration.
The errors that are present on the boards can be either easily fixed or an acceptable work-around has been found which can be used until a second iteration of the PCBs is designed.

Curiously, the \texttt{nRF24L01} module, perhaps the least analysed part of the project, is also the part that is least functional.
It was found that it had a far larger protocol than initially assumed and was also mounted incorrectly.
\\~\\
The intention of the authors was to design a real-time software system based on KHAos to show that the system is functioning as desired.
As has been explained previously, this goal was unfortunately not met.
Small pieces of software have been written to verify the functionality of individual components but a combined system was never written due to time constraints.

A number of features were implemented in VHDL to leverage the FPGA in the Zynq chip.
The authors purposefully implemented every VHDL component with an accompanying testbench before packaging it as an IP core.
This allowed for faster iterations when debugging VHDL code.
\\~\\
Data was gathered from the joints and is presented in the report.
A number of difficulties arose with the debugging of the full-bridge, but the issue, or at least a symptom of it, was eventually tracked down and a workaround determined.
The motor of the cart system has been driven by the designed motor driver.

Current measurement was not accomplished due to the noisy environment of the amplifier of the current signal.