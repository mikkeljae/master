%!TEX root = ../main.tex
\subsection{Implementation and Verification} % (fold)
\label{sub:implementation_and_verification}
This section will elaborate on the implementation of the design created in section \ref{sub:analysis}.
Specifically the methods and equipment used is discussed.
In addition, the verification of the finished product is presented in appropriate detail.
This will give the reader an impression of the method used to verify the implementation.

\subsubsection{Verification Methodology} % (fold)
\label{ssub:testing_methodology}
From experience of the authors, verification of a PCB design can become difficult without a predefined plan, increasing the risk of mistakes, oversights and similar problems.
In order to minimize the risk of these, it was decided to create a verification method prior to the actual verification of the PCB.
Each verification procedure is written to reflect the features created in the design.
Each of these features should be tested, in the intended order.
It should be written such that the verification can be done with no prior knowledge of the operation of the design.
This approach holds a few different benefits.
Firstly, the technician doing the verification is doing less thinking while verifying, this is likely to decrease the number of mistakes.
Secondly, a thorough verification procedure will enable students to more easily reproduce the PCB's if necessary.\\~\\
Each procedure is presented in the report with the results as gathered by the authors.
A version suitable for future use can be found in appendixes 
% subsubsection testing_methodology (end)

\subsubsection{Joint Board Verification Procedure} % (fold)
\label{ssub:joint_board_verification_methodology}
This section serves as the verification procedure for the joint board.
Each step is accompanied
% subsubsection joint_board_verification_methodology (end)
% subsection implementation_and_verification (end)