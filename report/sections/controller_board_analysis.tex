%!TEX root = ../main.tex
\subsection{Analysis}
\label{sub:controller_board_analysis}

In order to develop a controller board capable of controlling a double pendulum, the required functionality of the board needs to be determined. 
This section will constitute an analysis of this functionality and an analysis of possible solutions to desired functionality.

\subsubsection{Motor Driver}
It is necessary to device some form of driving circuitry for the motor driving the timing belt.
The motor should be able to move in both directions. 
An H-bridge allows this bidirectional movement.
By properly switching the MOSFETs in the bridge the average voltage across the motor can be controlled and therefore also the speed and direction of the motor.
As such, an H-bridge must be designed for the controller board.
\\~\\
An H-bridge of the size likely required here will require a sizeable capacitor bank across the supply near the H-bridge.
This is done to protect the remainder of the PCB against voltage spikes generated by parasitic inductances. 
They also serve to supply any sudden power requirements by the motor.
The necessary amount of supply capacitance must be determined.
\\~\\
When introducing a large amount of capacitance in a circuit, a problem known as inrush arises.
This happens when the circuit is powered on and the initial charging of the capacitors occur.
Due to the amount of capacitance, the circuit will draw a very large current for a significant amount of time which can potentially destroy parts of the circuit.
For this reason it is necessary to design circuitry which will temporarily limit the current through the circuit until the supply capacitors are charged.
Such a circuit is often comprised of a relay in series with a resistor.
The design of the circuitry surrounding the inrush relay must be designed.

\subsubsection{Voltage Rails}
The controller board will have a variety of electronics and there will be a need for different voltage rails as standard digital electronics cannot use the 24V rail that is used to drive the motor.
Each voltage rail needs to be designed to be able to deliver the power required by the components present on that rail.

\subsubsection{Motor Current Sensing}
In order to implement current mode control it is necessary to accurately measure the current being drawn by the motor at any given time.
Current measurements are generally done by either hall effect based sensors or shunt resistors. 
Hall effect based sensors are usually better for high current applications where adding a resistor in series with the load is infeasible. 
Additionally, their interference with the circuit they are measuring is negligible.
Hall effect sensors however, do come at a significantly higher cost.
A shunt resistor used for current measurements is simply a resistor with a very low resistance, typically $<$ 100m$\Omega$, and a high power rating ranging from a few watts into 10's of watts depending on the application.
By adding such a resistor in series with the load, the voltage across the resistor can be measured and from that the current through the load determined.
\thomas{What are the requirements of the accuracy? What can we achieve with the hardware?}


\subsubsection{Safety} % (fold)
\label{subsub:safety}
It is expected that, eventually, at an error may occur that causes the cart to move uncontrollably.
A safety system should be constructed in such a way that it will prevent the cart from crashing violently.
It should be isolated from the remaining system.
Therefore, if it can at all be avoided, no programming should be involved in determining a safety condition.
With this in mind, the safety system for this platform should be designed in such a way that, when activated, it will cut power to the motors.
In order to more easily identify the cause of the fault, the remaining electronics should remain operational to maintain the current program status.
Clearly, the system should function only if all safety conditions are met.
This will be ensured by creating an aggregation circuit which compares all of the safety conditions, creating one signal which will be capable of closing the main relay if necessary.
\\~\\
\mikkel{Endstops should be explained earlier in the report}
One of the safety features of the system is the endstops.
These are in place to quickly cut power to the system if the cart reaches the end of the rail.
Creating these endstops can be done in different ways.
Three potential implementations are presented here:
\paragraph{Mechanical Switch:} % (fold)
\label{par:mechanical_switch}
The simplest form of switch is the mechanical switch.
The switch should be mounted in such a way that the cart would move into the switch, therefore activating it.
This approach is not without issues.
Mounting of the switch should be done in such a way that it is not in the direct path of the cart.
If the cart is traveling at full speed it is unlikely that the cart would stop before crashing into the switching mechanism.
A switch mounted in this fashion would need to be rather robust.
Some other mounting solutions could be thought of that do not suffer from this problem, especially if a flexible microswitch is used.
These would require significant extra design to properly mount the switch.
Mechanical switches can be fragile and may not be sufficiently durable, considering the usecase.
All mechanical switches have one drawback in common: they are mechanical and will be subject to wear over time, eventually requiring maintainance or replacement.

% paragraph mechanical_switch (end)
\paragraph{Hall Effect Sensor:} % (fold)
\label{par:hall_effect_sensor}
This type of sensor measures magnetic fields and produces a voltage proportional to the strength of the field.
By placing a small neodymium magnet on the cart and mounting a hall effect sensor on the rail in such a way that the coincide would allow for detecting when the cart is above the sensor.
If the sensor is combined with a schmitt-trigger circuit the output could be a binary result, either the endstop is reached or it is not.
Using this method requires that a magnet is placed on the cart itself and that a sensor is mounted to the rail.
It should be noted that accelerating rather strong magnetic fields back and forth on the platform may not be the least electrically noisy solution one could think of.
Magnetic fields are also wide, meaning that the cart would "sneak up on" the threshold.
This requires some amount of calibration to determine the correct distance between sensor and magnet, which may not be mechanically simple to determine. 
% paragraph hall_effect_sensor (end)
\paragraph{Infrared Transceiver:} % (fold)
\label{par:infrared_transceiver}
Infrared transceivers come in different configurations but generally the device is comprised of an LED and a photo transistor.
The LED is optimised for emitting infrared light, or radiation (IR).
In the configuration under consideration in this project the LED emits IR towards the photo transistor, opening it.
When an object passes between the LED and the transistor the transistor is closed.
This change can be detected using a simple schmitt-trigger circuit. 
With this sensor type it is important to realise that the LED will not be the only emitter of IR in the vicinity.
Any type of lamp will emit IR, especially glowbulbs but also sunlight contain a significant amount of IR.
When using this sensor the circuit should be designed in such a way that only the effect of the IR from the LED on the photo transistor will affect the circuit.
% paragraph infrared_transceiver (end)
\\~\\
While all of the above have their drawbacks, it was decided to use the infrared transceiver.
This sensor allows the greatest reliability while being reasonably simple to mount on the platform in that it requires no modification to the cart or rail.
\\~\\
In addition to the endstops, also an emergency button, preferably red, should be able to cut power to the motors.

\subsubsection{Board Layout and Considerations} % (fold)
\label{ssub:board_layout_and_considerations}
A number of options are available when considering the general layout of the controller circuitry.
Different circuits can be segregated into individual board, simplifying the layout and production of each board but imposing requirements on board-to-board connections between different circuits.
One possible configuration under consideration is the segregation of the digital circuit and the power circuits.
This would allow a different number of layers and different copper thickness on the two PCBs.
Extra layers are beneficial when routing the multitude of signals in a digital circuit but also add extra cost to the production of the circuit.
Similarly, extra copper is likely to be necessary on the power circuit but also adds cost.
Exploring various manufacturing options revealed that the added cost of making a new design significantly outweighs the savings by segregating the thicker copper and extra layers, rather than producing just a single PCB incorporating both extra layers and extra copper.
For this reason it was decided to produce just a single PCB, housing all of the electronics related to the controller.
\\~\\
In summary, it was found necessary to implement the following:
\begin{itemize}
	\item H-bridge using MOSFETs to drive the motor.
	\item Supply capacitors and appropriately sized inrush relay.
	\item Voltage rails for the various components in the circuit.
	\item Shunt resistor circuit for correctly sensing current.
	\item Endstops and appertaining circuitry.
	\item Emergency button and appertaining circuitry.
	\item Security aggregation circuitry.
\end{itemize}
% subsubsection board_layout (end)