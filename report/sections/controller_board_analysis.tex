%!TEX root = ../main.tex
\subsection{Analysis}
\label{sub:controller_board_analysis}

In an effort to split the development into smaller, more managable parts, it was decided to initially develop the system such that it fulfills the following requirements:

\begin{itemize}
	\item The cart should be able to move with known position.
	\item Endstops on the platform should stop movement.
	\item Emergency button should stop movement.
\end{itemize}

Clearly, making the cart move is crucial in the development of the system and forms the basis for any later developments.
The latter two requirements are related to the safe operation of the platform.
It is expected that at least during development, an error may occur that causes the cart to move uncontrollably.
In such a situation the programmer should be able to completely shut off power to the motor.
The endstops will ensure that a minimum of mechanical or electrical damage is incurred on the platform in case of a cart crash.
The following paragraphs will explore what is required in order to fulfill the above requirements.
\subsubsection{Safety Circuitry} % (fold)
\label{subsub:safety_circuitry}
Safety first.
A safety system should, whenever possible, be isolated from the remaining system.
That is, it should depend on non of the remaining circuitry or programming.
Usually, if it can at all be avoided, no programming should be involved in determining a safety condition.
With this in mind, the safety system for this platform should be designed in such a way that, when activated, it will cut power to the motors.
In order to more easily identify the cause of the fault, the remaining electronics should remain operational to maintain the current program status.
\\~\\
Creating the endstops can be done in a multitude of ways.
In this project three approaches were considered:
\paragraph{Mechanical Switch:} % (fold)
\label{par:mechanical_switch}
The simplest form of switch is the mechanical switch.
The switch should be mounted in such a way that the cart would move into the switch, therefore activating it.
This approach is not without issues.
Firstly, the simplest mounting solution would require the switch to be mounted in the direct path of the cart.
If the cart is traveling at full speed it is unlikely that the cart would stop before crashing into the switching mechanism, potentially damaging it.
A switch mounted in this fashion would need to be rather robust.
Some other mounting solutions could be thought of that do not suffer from this problem, especially if a flexible microswitch is used.
These types of switches can be fragile and may not be sufficiently durable, considering the usecase.
All mechanical switches have one drawback in common: they are mechanical.
Generally, mechanical items wear out over time and require maintainance or replacement.
% paragraph mechanical_switch (end)
\paragraph{Hall Effect Sensor:} % (fold)
\label{par:hall_effect_sensor}
This type of sensor measures magnetic fields and produces a voltage proportional to the strength of the field.
By placing a small neodymium magnet on the cart and mounting a hall effect sensor on the rail in such a way that the coincide would allow for detecting when the cart is above the sensor.
If the sensor is combined with a schmitt-trigger circuit the output could be a binary result, either the endstop is reached or it is not.
Using this method requires that a magnet is placed on the cart itself and that a sensor is mounted to the rail.
It should be noted that accelerating rather strong magnetic fields back and forth on the platform may not be the least electrically noisy solution one could think of.
Magnetic fields are also wide, meaning that the cart would "sneak up on" the threshold.
This requires some amount of calibration to determine the correct distance between sensor and magnet, which may not be mechanically simple to determine. 
% paragraph hall_effect_sensor (end)
\paragraph{Infrared Transceiver:} % (fold)
\label{par:infrared_transceiver}
This type of device is a combined LED and infrared sensor.
The LED is optimised for emitting infrared light, or radiation (IR), and the sensor for sensing it.
The LED emits IR outward from the device which, when an object is placed in front of it, will bounce off the object and be caught by the sensor.
As with the hall effect sensor, this type of device produces a voltage output proportional to the amount of IR being sensed and as such a schmitt-trigger circuit would also be benficial in this case.
With this sensor type it is important to realise that the LED will not be the only emitter of IR in the vicinity.
Any type of lamp will emit IR, especially glowbulbs (of which there are few left, luckily) but also sunlight contains some amount of IR.
When using this sensor the circuit should be designed in such a way that only the effect of the IR radiated back on the sensor will trigger the circuit.
The sensor should be mounted immediately below the cart so as to maximize the radiated IR and therefore the signal strength.
% paragraph infrared_transceiver (end)
\\~\\
While all of the above have their drawbacks, it was decided to use the infrared transceiver.
This sensor allows the greatest reliability while being reasonably simple to mount on the platform in that it requires no modification to the cart.
\\~\\
In addition to the endstops, also an emergency button, preferably red, should be able to cut power to the motors.
It may be that additional safety features are added later.
In order to minimize the amount of circuitry, all safety features will trip the same circuit.
\thomas{Add circuit showing the safety relay}
On figure \ref{fig:relay_circuit} is shown the safety circuitry.
A relay is mounted in series with the supply rail for the motor.
The default state for this relay should be \texttt{off}.
Then, in order to switch the system to the \texttt{on} state, the relay should be closed.
Should power to the relay coil ever fail i.e. something in the system has broken, power to the motor is cut off.
Since the relay is in series with the motor, it should also be capable of carrying the full design current, 80A.
One relay that fulfills these requirements is the.
\\~\\
The driver circuit for the relay is activated only when all safety conditions are off.
If at any point any of the safety conditions are tripped, the driver circuit shuts down, releasing the relay and turning off power to the motor.
Designing the safety circuit in this manner ensures that the system cannot be started or will shut down if any wires or components in the safety circuit break.
