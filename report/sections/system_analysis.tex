%!TEX root = ../main.tex
\section{State of the Art}
The system developed throughout this report is a proposal made by the project owner and developed by the authors.
It is intended as a learning platform for students to experiment with and gain understanding of underactuated systems.
An underactuated system is a system with more DoF (degrees of freedom) than actuators.
Traditionally in control theory, an equal or greater number of actuators than DoF are used to cancel out the dynamics of a system.
Underactuated systems is a vast field of research including examples in personal transportation such as the self-balancing scooter \cite{scooter} or the Segway \cite{segway}.
Both of these systems are essentially practical implementations of the inverted pendulum where an on-board control system will attempt to keep the device upright while maintaining a steady speed based on the tilt of the device.
Also underactuated robotic grippers are seen in abundance in literature. 
\cite{threegripper} is an implementation of a three-phalanx finger gripper which utilises two underactuated fingers.
When the fingers are brought together around an object, using the only actuated degree of freedom, the fingers wrap themselves around the object, regardless of size or shape.
This means that a gripper can be far more versatile and much less dependent on the correct placement of the object. 
\cite{pullgripper} implements a two-to-five finger gripper with belts on each finger to pull objects into the grasp of the gripper.
\\~\\
In control of underactuated systems it is required to exploit the natural dynamics of the system \cite{mitunderactuated} in order to gain full control.
Such systems are generally more agile and efficient than their fully actuated counter-parts.
\\~\\
A general example often used for teaching underactuated systems is the inverted pendulum.
Numerous examples of such systems are described in literature, but three versions in particular are presented here to provide an overview:
\begin{itemize}
	\item The pendubot \cite{pendubot} is a pendulum system consisting of one unactuated pendulum mounted on the end of an actuated pendulum.
	\item A rotary inverted pendulum \cite{rotarypendulum} mounts an unactuated pendulum on a crossbar which is mounted horizontally on the axle of an upright DC-motor.
	\item The cart mounted inverted pendulum \cite{invertpendulum3} actuates a cart along a rail in order to generate motion in the pendulum.
\end{itemize} 
In common for all of the above systems is that the joint between the mounting platform and the platform is implements some form of encoder.
This allows knowing the position of the pendulum and thereby controlling the system.
The pendubot and the rotary inverted pendulum have the feature that they can control indefinetely without running out of space, as is the case with the cart-on-a-rail system.
\\~\\
The system under development throughout this report is of the cart-on-a-rail type.
Despite the apparent short-coming of limited control range, this version of the inverted pendulum system is quite popular in literature.
\cite{invertpendulum3} is an implementation of a single inverted pendulum while versions with series pendulums exist as well such as the double pendulum shown in \cite{doubleinvertpendulum} or the triple pendulum shown in \cite{tripleinvertpendulum}.
\\~\\
Both \cite{doubleinvertpendulum} and \cite{tripleinvertpendulum} utilize encoders with a resolution of 8192 counts per revolution, or $\approx 0.044^\circ$.
\cite{invertpendulum3} is implemented using just 5000 counts per revolution, or $\approx 0.072^\circ$.
In addition to pure positional control by encoders, some systems \cite{invertpendulum1} also use current-mode control, CMC.
CMC is a method whereby the voltage across a system can be controlled and, in the case of a motor, therefore also the velocity of the motor.
\\~\\
\cite{doubleinvertpendulum} and \cite{tripleinvertpendulum} both use a 1ms sampling time.
The underlying system is different between the two with \cite{tripleinvertpendulum} being based on DSpace and \cite{doubleinvertpendulum} a linux system with the real-time patch applied.
\\~\\
All of the literature mentioned above all utilize the Euler-Lagrange equation of motion for developing the mathematical model required to control their respective pendulum systems. 
These equations assume point masses and massless links between them.
The authors have derived these equations for the double pendulum in appendix \ref{sec:model_development}.
\thomas{a note on its usefulness?}

\section{System Analysis}
This document is a development report detailing the creation of such a system.
The authors have taken inspiration from the aforementioned literature to develop requirements and implementation ideas.
\\~\\
Throughout the remainder of this section an analysis of the system will be done and the overall design requirements of the system determined.
In common for all of the cited systems is that they consist of a cart capable of moving along a rail with a pendulum mounted to it.
The cited systems include single inverted pendulums \cite{invertpendulum1} \cite{invertpendulum3}, a double pendulum \cite{doubleinvertpendulum} and a triple pendulum \cite{tripleinvertpendulum}.
This project will aim to create a double pendulum system.
By actuating only the cart, the pendulums will not be directly actuated and the entire system therefore underactuated.
The connection-point between the cart and the first pendulum and between every consecutive pendulum should be a joint capable of rotating freely.
Performing swing-up of the pendulum assembly requires knowledge of both the cart position on the rail and the position of every joint in the pendulum assembly.  
\\~\\
The designed system should be a teaching platform first and foremost.
This means that a number of users with varying levels of understanding of the underlying hardware are likely to handle the system throughout its lifetime.
Therefore the system should be simple and safe to use.
The exact definition of simple will clearly vary with different users but users are generally expected to have a working understanding of Linux.
For this reason the project owner has requested that the main computation platform is based on the Xilinx Zynq-7 series architechture which provides the possibility of running Linux on the internal ARM cores while still maintaining the speed and flexibility of an FPGA.
The authors have prior knowledge of the use and design of systems with the MicroZed development board and will therefore utilize this platform in the development of this project. 
The original intent was to develop and implement a kernel driver which would present the system to the user as a \texttt{tty} interface, however this quickly turned infeasible due to the time constraints of the project and will be left for future projects.
Further design of the system should still maintain focus on the eventual goal of making the system accessible through Linux.
\\~\\
Safety around the system is important since the movement of a double pendulum is chaotic and therefore inherently unpredictable.
A means of disabling power to the motor in case of emergency should be present.  
Engaging an emergency condition should be possible both manually and automatically.
The handling of emergency conditions should happen independently of any programming so that faulty programming cannot disable handling.
End stops should be mounted on the rail to provide a means of generating an emergency condition.
This is done to avoid damage to the system in case of a run-away cart.
It is likely that the system should remain caged once finished so as to ensure that no person is in the path of the pendulum during run-time but the requirements to such a safety system are very much dependent on the system around which it should be developed and will be left as a future project to be handled before the pendulum system is given to a user.
In addition to the above, the system should also be tolerable to some amount of misuse such as blocking of the cart and similar unexpected situations.
\\~\\
Since the focus of the authors is electronics and embedded systems design, this will also be the main focus of this project.
Therefore the design of electronics will be done by the authors. 
The electronics include a board for motor control and main computation as well as a board for interfacing the incoder in the joints, the specifics of which are discussed throughout the report.
\\~\\
Since creating the final version of the software is infeasible given the time frame, another solution is required to validate the functionality of the system.
For this reason a real-time software system should be developed which correctly exercises every part of the developed pendulum system.
Additionally the distribution of functionality between the FPGA and the ARM cores should be determined to optimally make use of their abilities.
The overall software design should be determined prior to initiating programming.
Programming of individual components such as drivers for sensory equipment can be done bottom-up and integrated into a larger context once confirmed functioning.
\\~\\
Early in the process the authors inherited the rail and cart system from a previous project.
This includes a Maxon 148867 motor with an included HEDS5540 encoder, both of which will be explored further in later sections.
The rail system is used as-is since it is expected that any shortcomings of the rail system in relation to the double pendulum application will be revealed in using the finished system.
An evaluation of the usefulness of the rail system should be done on the finished system.

\subsection{Requirement Specification}
\label{sub:requirements}
The requirements specified below are tested and verified in section \ref{sub:verification_of_requirement_enum:system_accessible_to_users}.

\paragraph{Functional:}
\begin{enumerate}	
	\item System should consists of double pendulum mounted on a moveable cart.
	\label{enum:system_consists_double_pendulum}
	\item Cart should be actuated by a motor. 
	\label{enum:cart_actuated_by_motor}
	\item The pendulum system should be controlled by a Microzed.
	\label{enum:controlled_by_microzed}
	\item Position of the cart and joint angles should be measured.
	\label{enum:position_cart_joint_angle_measure}
	\item Any software developed for the pendulum system should be real-time.
	\label{enum:software_developed_real_time}
	\item System should be easily accessible to users.
	\label{enum:system_accessible_to_users}
\end{enumerate}

\paragraph{Safety:}
\begin{enumerate}[resume]
	\item System should not break during operation by users.
	\label{enum:system_should_not_break_during_operation}
	\item System safety should not rely on the programming of the system.
	\label{enum:safety_should_not_rely_on_programming}
\end{enumerate}

\paragraph{Design:}
\begin{enumerate}[resume]
	 \item Software design is done top-down, whereas subcomponents are written bottom-up. 
	 \label{enum:software_top_down_bottom_up}
\end{enumerate}