%!TEX root = ../main.tex
\section{System Analysis}
The system developed throughout this report is a proposal made by the project owner and developed by the authors.
It is intended as a learning platform for students to experiment with and understand underactuated systems.
Underactuated systems is a vast field of research including examples in personal transportation \cite{scooter} \cite{segway} and robotic grippers \cite{threegripper} \cite{softgripper}.
An underactuated system is a system with more DoF (degrees of freedom) than actuators.
Traditionally in control theory, an equal or greater number of actuators than DoF are used to cancel out the dynamics of a system.
In control of underactuated systems it is required to exploit the natural dynamics of the system \cite{mitunderactuated} in order to gain full control.

\subsection{Requirement Specification}
\label{sub:requirements}

\mikkel{The following is not finished, but a sketch of the requirements.}
\paragraph{Functional:}
\begin{enumerate}	\item System should consists of double pendulum mounted\\ on a moveable cart.\hfill S. 3.3
	\item Cart should be actuated by a motor. \hfill 2.3
	\item The motor should be controlled by a Microzed.\hfill 2.1
	\item Position of the cart and joint angles should be measured.
	\item Software on Microzed and joints should be real time.
	\item System should be simple to use
	\item Joint board and controller board.........
\end{enumerate}

\paragraph{Safety:}
\begin{enumerate}[resume]
	\item System should not break when misused.
	\item System safety should not rely on users programming.
\end{enumerate}

\paragraph{Design:}
\begin{enumerate}[resume]
	\item All parts of the system should be developed using the following method. 
	\begin{itemize}
	 	\item Analysis.
	 	\item Requirement Specification.
	 	\item Design.
	 	\item Implementation.
	 	\item Verification.
	 \end{itemize} 
	 \item System should be designed with learning outcome in mind (not minimizing cost).
	 \item All software should be developed by making a top-analysis initially and hereafter working bottom-up on small parts. 
\end{enumerate}