%!TEX root = ../main.tex
\section{Controller Board Verification Procedure} % (fold)
\label{sec:controller_board_verification_procedure}
\thomas{add endstop test}
This document holds the verification procedure which, when applied succesfully, ensures the correct functionality of the controller board.
Each step is to be followed in the order they are shown to avoid previous faults to propegate throughout the process.
A step consists of an action and the expected outcome of that action.
Any text written with the \texttt{teletype font} refers to signals and pins on the PCB and can be seen on the schematic.
The most recent schematic can be seen on appendix \ref{app:controller_board_schem}.
\paragraph{Verify Voltage Rails} % (fold)
 \begin{itemize}
 	\item Apply 24V to \texttt{24V}.
 	\item Toggle power button to ON.
 	\begin{itemize}
 		\item[-] Verify that current draw is within reasonable limits ($<$150 mA).
 		\item[-] Verify 12V, 5V, 3.3V and 2.5V voltage rails.
 	\end{itemize}
 \end{itemize}

\paragraph{Simulate No Emergency} % (fold)
\begin{itemize}
	\item Apply 5V to \texttt{EM\_END1} and \texttt{EM\_END2}.
	\begin{itemize}
		\item Verify that 0V is present on \texttt{EM\_END1\_OUT} and \texttt{EM\_END2\_OUT}.
	\end{itemize}
	\item Apply 0V to \texttt{EM\_END1} and \texttt{EM\_END2}.
	\begin{itemize}
		\item Verify that 5V is present on \texttt{EM\_END1\_OUT} and \texttt{EM\_END2\_OUT}.
	\end{itemize}
	\item Set signals \texttt{EM\_END1}, \texttt{EM\_END2} and \texttt{EM\_MCU} to 0V.
	\item Ensure \texttt{EM\_BTN} is set to 5V.
	\begin{itemize}
		\item Verify that 5V is present on \texttt{EM\_DIS}
	\end{itemize}
\end{itemize}
% paragraph simulate_no_emergency (end)
\paragraph{Simulate Emergency} % (fold)
\begin{itemize}
	\item Set signals \texttt{EM\_END1}, \texttt{EM\_END2} and \texttt{EM\_MCU} to 0V.
	\item Ensure \texttt{EM\_BTN} is set to 5V.
	\item For each signal, toggle to either 5V or 0V.
	\begin{itemize}
		\item Upon toggling one signal, verify that 0V is present on \texttt{EM\_DIS}.
	\end{itemize}
\end{itemize}
% paragraph simulate_no_emergency (end)
\paragraph{Verify Relay Circuitry Functionality} % (fold)
\begin{itemize}
	\item Following the previous step, make \texttt{EM\_DIS} high.
	\item Apply 0V to \texttt{INRUSH}.
	\begin{itemize}
		\item Verify that 0V is present on the output of U5.
	\end{itemize}
	\item Apply 5V to \texttt{INRUSH}.
	\begin{itemize}
		\item Verify that 5V is present on the output of U5.
		\item Verify that 24V is present on \texttt{POWER\_IN}
	\end{itemize}
	\item Apply 0V to \texttt{M\_RELAY}.
	\begin{itemize}
		\item Verify that 0V is present on the output of U6.
	\end{itemize}
	\item Apply 5V to \texttt{M\_RELAY}.
	\begin{itemize}
		\item Verify that 5V is present on the output of U6.
	\end{itemize}
	\item Apply 0V to \texttt{INRUSH}.
	\begin{itemize}
		\item Ensure that 24V is still present on \texttt{POWER\_IN}.
	\end{itemize}
\end{itemize}
% paragraph verify_relay_circuitry (end)
\paragraph{Verify Motor Driver Functionality} % (fold)
\begin{itemize}
	\item Apply 5V to \texttt{DIS}.
	\item Apply 3.3V PWM to signals \texttt{PWMA} and \texttt{PWMB}.
	\begin{itemize}
		\item Verify corresponding 5V PWM on signals \texttt{PWM\_AL} and \texttt{PWM\_BL}.
		\item Verify that 0V is present on signals \texttt{MGATE\_BH}, \texttt{MGATE\_BL}, \texttt{MGATE\_AH} and \texttt{MGATE\_AL}
	\end{itemize}
	\item Apply 0V to \texttt{DIS}.
	\item Ensure that 24V is still present on \texttt{POWER\_IN}.
	\begin{itemize}
		\item Verify that 12V PWM is present on signals \texttt{MGATE\_AL} and \texttt{MGATE\_BL}.
		\item Verify that 35V PWM is present on signals \texttt{MGATE\_AH} and \texttt{MGATE\_BH}.
		\item Verify that deadtime exists between the two channels.
	\end{itemize}
\end{itemize}
% paragraph verify_motor_driver_functionality (end)
\paragraph{Verify nRF24L01 Module functionality} % (fold)
\begin{itemize}
	\item Apply 3.3V to signals \texttt{RF\_MISO}, \texttt{RF\_MOSI}, \texttt{RF\_SCK}, \texttt{RF\_CSN} and \texttt{RF\_CE}.
	\begin{itemize}
		\item Verify that 3.3V is present on the corresponding signals on the \texttt{nRFM} module (component U1).
	\end{itemize}
\end{itemize}

\paragraph{Verify Shunt Amplifier Functionality} % (fold)
\label{par:verify_shunt_amplifier_functionality}
\thomas{shunt amplifier section should be populated.}
